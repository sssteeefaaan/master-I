\chapter{Закључак}
У овом раду се детаљније размотрио проблем проналажења најкраћег пута, односно најмањег броја скокова, између два чвора у графу и то применом алгоритма за претрагу графа по ширини, познатог као \textit{breadth-first traversal}. Имплементирани су алгоритми паралелнe верзијe овог алгоритма, користећи системе са дистрибуираном и дељеном меморијом. Takoђе, дат је и приказ резултата добијених примењујући поменуте имеплементације на насумично генерисаним графовима.

\par
У првом одељку су дате теоријске основе графова. Обрађена је основна терминологија, идеје основних операција над њима, описане су идеје претраживања истих итд.. Такође, у оквиру овог одељка су описани и разни начини репрезентације графа, поједини од којих су коришћени ради олакшане обраде у паралелном окружењу.

\par
У оквиру другог одељка су изнети описи појединих приступа у паралелизацији претраживања графа. Поред описа, ово поглавље садржи и идеје имплементирања сваког од појединих приступа, заједно са алгоритмима и описом истих. 

\par
Треће поглавље садржи имплементацију алгоритама за паралелно претраживање графа у дистрибуираним и системима са дељеном меморијом. Такође, на крају описа сваке од њих стоји приказ резултата добијених кроз њихову примену на насумично генерисаним графовима.

\par
На основу анализе имплементација, може се закључити да је реализација паралелног претраживања графа могућа, али ни под разном једноставна и никако апсолутна. Убрзања до којих се дошло су ту искључуво јер се ради о вештачки синтезованим графовима, као и чињеница да та убрзања постоје само за случајеве високог степена повезаности, односно паралелизације у оквиру окружења које ради са дељивом меморијом.

\par
У оквиру секције која описује идеје које стоје иза паралелизације обраде графа по ширини су описани и методи који нису имплементирани у овом раду, а могуће је да би дали добре резултате. Њихова имплементација може бити корак ближе генералном решењу овог проблема, тако да се читаоци овог рада охрабрују да исту покушају.