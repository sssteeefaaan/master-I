\section{Креирање насумичног графа}
Имплементација функције која врши креирање насумичног графа је приказана на исечку \ref{code:generate-random-graph}. Фукција \cinline{void generate_random_graph(...)} прима параметре:
\begin{enumerate}
    \item \mintinline{c}{long long** graph} - показивач на низ целобројних вредности у оквиру кога ће бити бележени сегменти са суседима за одговарајуће теме,
    \item \mintinline{c}{long long* degrees} - низ целебројних вредности који ће бележити помераје у оквиру низа \mintinline{c}{graph}, како би се знало који сегмент низа припада ком темену,
    \item \mintinline{c}{long long vertex_numb} - целобројна променљива која представља број чворова у графу,
    \item \mintinline{c}{long long max_degrees} - целобројна променљива која представља максимални број потега који једно теме може да има и
    \item \mintinline{c}{long long min_degrees} - целобројна променљива која представља минимални број потега који једно теме може да има.
\end{enumerate}
На почетку (у оквиру линија 179 и 180) се решавају граничне вредности променљивих \mintinline{c}{max_degrees} и \mintinline{c}{min_degrees}, док се у оквиру линије 181 врши иницијализација целобројне промељиве \mintinline{c}{start} на вредност 0, што ће бити померај првог темена у графу и кроз сваку итерацију ће се вредност ове промељиве акумулирати и бележити у низ помераја \mintinline{c}{degrees}. То се дешава у линијама 183-187, где се у петљи генерише насумична вредност у опсегу $[max\_degrees-min\_degrees]$ и додаје на претходну вредност променљиве \mintinline{c}{start}.

\par
У линији 189 се врши алоцирање меморије за сам низ потега \mintinline{c}{graph}, док се у линијама 192-197 врши и сама иницијализација, односно генерисање насумичних потега. Кроз две $for$ петља се пролази кроз низ \mintinline{c}{graph}, читају вредности у оквиру низа \mintinline{c}{degrees} и генеришу уникатне вредности у опсегу $[0-vertex\_numb)$. Фуникција \cinline{void reset_options(...)} врши иницијализацију низа на опсег могућих вредности, док функција \cinline{long long get_unique_random(...)} из низа извлачи елемент са насумичне позиције, тај елемент враћа, а на његово место убацује последњи елемент из низа, што за резултат има избацивање резултујуће вредности из низа.

\begin{listing}[H]
\inputminted[fontsize={\fontsize{7}{6}\selectfont},firstline=160,lastline=202]{c}{kodovi/bfs-dist-general-1d.c}
\caption{\textit{Генерисање насумичног графа}}
\label{code:generate-random-graph}
\end{listing}