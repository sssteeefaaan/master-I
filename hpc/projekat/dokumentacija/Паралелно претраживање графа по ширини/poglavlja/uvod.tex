\chapter*{Увод}
Употреба апстракције уз помоћ графова за анализу и разумевање разних врста података добија све већи значај. Неки од примера података који могу да се апстрахују користећи графове подразумају: податке о интеракцијама на друштвеним мрежама, податке банкарских трансакција, податке о препоруци разних реклама корисницима апликација на основу њихових интеракција, комуникационих података попут електронске поште и телефонских мрежа, податке биолошких система и различитих облика релационих података генерално. Када се говори о вештачкој интелигенцији, апсолутно је неопходно увести неку врсту графа и примењивати разноврсне алгоритме над њим. Заједнички проблеми у математичкој области теорије графова и у областима примене укључују идентификацију и рангирање важних ентитета, откривање аномалија у обрасцима или изненадних промена у мрежама, проналажење чврсто повезаних кластера ентитета, итд. Решења ових проблема обично укључују класичне алгоритме над графовима, као што су: пребацивање графа у структуру стабла (уклањање циклуса), проналажење најкраћих путања, проналажење двоповезаних компоненти, упаривања, прорачуне засноване на протоку итд. Да бисмо задовољили потребе теоријске анализе графова за нове апликације које захтевају рад са структурама великих скупова података, од суштинског је значаја да убрзамо основне проблеме графова користећи актуелне паралелне системе \cite{graphs-big-data}.

\par
У овом раду ће се детаљније разматрати проблем проналажења најкраћег пута, односно најмањег броја скокова, између два чвора у графу и то применом алгоритма за претрагу графа по ширини, познатог као \textit{breadth-first traversal}. Биће имплементирани алгоритми паралелнe верзијe овог алгоритма, користећи системе са дистрибуираном и дељеном меморијом. Takoђе, биће дат и приказ резултата добијених примењујући ове имплементације на насумично генерисаним графовима.

\par
У првом одељку ће се изнети теоријске основе графова, попут терминологија, основних идеја операција над њима, описом идеја претраживања истих и сл. Такође, у оквиру овог одељка ће бити описани и разни начини репрезентације графа, поједини од којих ће бити коришћени ради олакшане обраде у паралелном окружењу.

\par
У оквиру другог одељка ће бити описани поједини приступи у паралелизацији претраживања графа. Поред описа, ово поглавље ће садржати и идеје за имплементирање сваког од појединих приступа, заједно са алгоритмима и описом истих. 

\par
Треће поглавље ће садржати имплементацију алгоритама за паралелно претраживање графа у дистрибуираним и системима са дељеном меморијом. Такође, на крају описа сваке од имплементација стојаће приказ резултата добијених кроз њихову примену на насумично генерисаним графовима.

\par
На крају, у последњем одељку, биће изнет и коначан закључак самог рада.