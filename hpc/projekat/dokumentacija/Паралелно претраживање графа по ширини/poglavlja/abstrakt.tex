\begin{abstract}
\doublespacing
Тема овог рада јесте дистрибуирано односно паралелно претраживање графа по ширини. Мотивација за писање рада јесте пре свега истраживање начина за убрзање релативно основних опереција над веома битном структуром података у рачунарству, односно графом. Такође, дубље истраживање проблема који се јавља у дистрибуираном и паралелном програмирању када се говори о претраживању графова. Наиме, због структуре која је, за недостатак бољих речи, непредвидива, тешко је добити прихватљиве перформансе при приступању нелинеарно складиштеним податацима. Кроз рад су анализиране методе за расподелу посла између процеса односно нити који врше саму претрагу, предности и мане оба приступа, као и постигнути резултати при њиховој примени. Закључено је да иако наизглед комплексно и мукотрпно, ипак је могуће постигнути неко убрзање кроз паралелни приступ у окружењима са дељеном меморијом. Нажалост, што се овог рада тиче, имплементација у оквиру окружења са дистрибуираном меморијом није показала позитивне резултате.
\end{abstract}
\pagebreak