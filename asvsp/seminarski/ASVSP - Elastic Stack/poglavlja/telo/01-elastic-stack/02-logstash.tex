\section{\textit{\textbf{Logstash}}}
\textit{\textbf{Logstash}} je alat otvorenog koda za prikupljanje podataka sa mogućnostima nadovezivanja \textit{(eng. pipelining)} u realnom vremenu. Alat može dinamičnim putem objediniti podatke iz različitih izvora, obogatiti ih, normalizovati i na kraju dostaviti podatke na unapred definisana odredišta. Ovim se pruža mogućnost prečišćavanja podataka za različite slučajeve upotrebe, naprednu analizu ili vizuelizaciju \cite{elastic-logstash}.

\par
Na početku je \textit{\textbf{Logstash}} bio korišćen za prikupljanje podataka evidencije \textit{(eng. log)}, međutim, mogućnosti koje pruža su prevazišle taj slučaj korišćenja. Naime, bilo koja vrsta događaja može biti prosleđena alatu, uz pomoć širokog spektra ulaznih dodataka \textit{(eng. input plugins)}, primljeni događaji mogu biti integrisani i transformisani uz pomoć dodataka za filtriranje \textit{(eng. filter plugins)} i na kraju tako formatirani podaci mogu biti prosleđeni na razne vrste izlaza, uz pomoć izlazih dodataka \textit{(eng. output plugins)}. Sam proces obrade je podržan velikim brojem često korišćenih kodera, a pritom je alat dizajniran za obradu velike količine podataka u realnom vremenu. Alat je razvijen u programskom jeziku JRuby i izvršava se na javinoj virtuelnoj mašini \textit{(eng. Java Virtual Machine)}, skraćeno \textit{JVM} \cite{jvm}.

\subsection{Kako \textit{\textbf{Logstash}} funkcioniše?}
\textit{\textbf{Logstash}} za sebe ima vezane cevovode, ili tokove podataka \textit{(eng. pipeline)} definisane ulaznim, transformacionim i izlaznim etapama. Ulazi očitavaju događaje, filteri ih modifikuju, a izlazi tako modifikovane događaje dostavljaju na definisana odredišta. Ulazi i izlazi imaju unapred podržane kodeke \footnote{Kodek \textit{(eng. codec)} kompresuje ili dekompresuje bilo koji sadržaj u i iz bajtova.\cite{codec}}, koji omogućavaju kodiranje i dekodiranje podataka pri ulazu i izlazu iz toka obrade, bez korišćenja transformacionih filtera. Primer jednog toka podataka prikazan je na slici \ref{diagram:primer-logstash-toka-podataka}.

\begin{figure}[H]
    \centering
    \fboxsep=0.025\columnwidth%padding thickness
    \fboxrule=1pt%border thickness
    \fbox{\includegraphics[width=0.95\columnwidth]{images/Logstash-pipeline.pdf}}
    \caption{\textit{Primer \textbf{Logstash} toka podataka}}
    \label{diagram:primer-logstash-toka-podataka}
\end{figure}

\par
Svaki dodatak koji se koristi u okviru ulazne etape se izvršava u sopstvenoj niti i svaki od njih upisuje događaje u centralizovani red čekanja koji je ili u radnoj memoriji pokrenutog procesa, ili može biti na disku. Svaki radnik toka podataka \textit{(eng. pipeline worker)} uzima blok događaja iz reda čekanja, sprovodi blok kroz definisane dodatke u okviru filter etape i na kraju šalje obrađene događaje na svaki od definisanih dodataka izlazne etape. Veličina bloka, kao i broj radnika toka podataka, odnosno niti, je podesiv kroz konfiguracione fajlove \textit{\textbf{Logstash}}-a.

\par
\textit{\textbf{Logstash}} podrazumevano koristi redove čekanja koji se nalaze u radnoj memoriji za baferovanje događaja između etapa, ulaz i filter etapa, odnosno filter i izlaz etapa, što je brže rešenje. Međutim pri nastanku greške, odnosno obustavi procesa, baferovani podaci bivaju izgubljeni.

\par
\textbf{Ulaznim dodacima} se definiše izvor podataka koji ulaze u sistem. Najčešće korišćeni ulazni dodaci podrazumevaju:
\begin{itemize}
\item \textbf{file} – vrši čitanje fajla sa fajl sistema nalik \textit{Unix} komandi \mintinline{shell-session}{tail -0F},
\item \textbf{syslog} – osluškuje na podrazumevanom portu 514, koji predstavlja port za \textit{syslog} poruke u \textit{Unix}-u i parsira ih na osnovu RFC3164 formata \cite{syslog-protocol},
\item \textbf{redis} – vrši čitanje sa \textit{Redis} servera, konkretno sa \textit{Redis} kanala i listi. \textit{Redis} \cite{redis} je često korišćen kao berzijanac poruka \textit{(eng. message broker)} u centralizovanoj instalaciji \textit{\textbf{Logstash}}-a i služi za skladištenje događaja u okviru reda čekanja,
\item \textbf{beats} – vrši obradu podataka koji se dobijaju uz pomoć \textit{Beats} alata \cite{elastic-beats}.
\end{itemize}

\par
textbf{Filter dodaci} predstavljaju procesne uređaje koji posreduju tokom podataka. Filteri mogu biti kombinovani uslovima koji diriguju da li se određena faza obrade izvršava ili ne, u zavisnosti od kriterijuma koji se ispituje. Najčešće korišćeni dodaci za filtriranje su:
\begin{itemize}
\item \textbf{grok} – parsira i gradi proizvoljan tekst. Ovaj dodatak je trenutno najpogodnije rešenje za parsiranje nestrukturnih tokova podataka u podatke koji imaju strukturu i moguće je nad njima izvršavati upite,
\item \textbf{mutate} – izvršava generalne transformacije nad poljima događaja, poput: preimenovanja, uklanjanja, zamene, modifikacije i slično,
\item \textbf{drop} – izvršava obustavu događaja u potpunosti. Korisno pri događajima koji imaju debug nivo ozbiljnosti,
\item \textbf{clone} – vrši replikaciju događaja, sa mogućnošću za dalje dodavanje i uklanjanje pojedinih polja,
\item \textbf{geoip} – nadograđuje \textit{IP} adresu geo-prostornim koordinatama, kao i dostupnim informacijama za iste.
\end{itemize}

\par
\textbf{Izlazni dodaci} predstavljaju destinacije konačne etape u izvršavanju toka podataka \textit{\textbf{Logstash}}-a. Događaj može proći kroz više izlaza, međutim, njegova obrada je završena kada prođe kroz sve navedene izlaze. Najčešće korišćeni dodaci za izlaz podrazumevaju:
\begin{itemize}
    \item \textbf{elasticsearch} – šalje događaj na \textit{Elasticsearch}, samim tim se vrši indeksiranje podataka koji dalje mogu lako da se pretražuju i vizualizuju u \textit{Kibani}.
    \item \textbf{stdout} – ispisuje događaj na standardni izlaz.
    \item \textbf{file} – skladišti formatiran događaj na fajl u fajl sistemu.
    \item \textbf{graphite} – šalje događaj na  \textit{graphite}, koji predstavlja popularan alat otvorenog koda za skladištenje podataka metrike. \cite{graphite}
    \item \textbf{statsd} – šalje podatke na  \textit{statsd}. Ovo je pozadinski servis  \textit{(eng. deamon)}, koji se izvršava na  \textit{Node.js} platformi, a čija je funkcija osluškivanje podataka statistike poslate  \textit{UDP} transportnim protokolom i prosleđivanje primljenih podataka na jedan ili više priključivih  \textit{(eng. pluggable)} pozadinskih  \textit{(eng. backend)} servisa, poput gore navedenog  \textit{graphite} servisa. \cite{statsd}
\end{itemize}

\subsection{\textit{\textbf{Logstash}} konfiguracioni fajlovi}
\textit{\textbf{Logstash}} sadrži dve vrste konfiguracionih fajlova, a to su:
\begin{itemize}
    \item \textbf{Pipeline konfiguracioni fajlovi}, koji definišu sam tok podataka. Izgled ovih fajlova će biti detaljnije objašnjen u nastavku.
    \item \textbf{Settings konfiguracioni fajlovi}, koji definišu inicijalizaciju i tok rada samog \textit{\textbf{Logstash}} procesa. Ovi fajlovi su generisani pri instaliranju samog \textit{\textbf{Logstash}} servisa i podrazumevaju fajlove:
    \begin{itemize}
        \item[o] \textbf{Logstash.yml} – sadrži konfiguracione indikatore \textit{(eng. flags)} za instancu programa. Ovde se mogu definisati: tip i lokacija bafera koji se koristi za red čekanja između etapa, veličina bloka događaja koju obrađuje jedan radnik, nivo evidentiranja \textit{(eng. logging level)} i mnoga druga podešavanja.
        \item[o] \textbf{Pipelines.yml} – sadrži konfiguraciju većeg broja tokova podataka koji će se izvršavati u okviru jedne instance \textit{\textbf{Logstash}}-a. Ovde se između ostalog navodi putanja do same konfiguracije toka podataka, kao i broj radnika u okviru toka, identifikator toka i slično.
        \item[o] \textbf{Jvm.options} – sadrži podešavanja za \textit{JVM} indikatore. Ovde se između ostalog može definisati minimalna i maksimalna veličina radne memorije dodeljene procesu.
        \item[o] \textbf{Log4j2.properties} – sadrži podrazumevana podešavanja za \textit{log4j} biblioteku \cite{log4j}.
    \end{itemize}
\end{itemize}

\subsubsection{Struktura konfiguracionih fajlova}
\textit{\textbf{Logstash}} konfiguracioni fajlovi za tokove podataka su pisani u specijalnom jeziku \cite{logstash-pipeline-language} koji je razvijen od strane samih osnivača programa. Jezik podseća na \textit{JSON} format sa par izuzetaka, ali je vrlo jednostavan za razumevanje.
Tok podataka, \textit{pipeline} ima odvojene segmente za ulaznu, filter i izlaznu etapu, čija je oblast važenja uokvirena vitičastim zagradama. U okviru svake od etapa se mogu navesti dodaci koji se koriste.  Bitno je naglasiti da se dodaci u okviru ulazne etape izvršavaju konkurentno, dok se u filter i izlaznoj etapi izvršavaju sekvencijalno.

\par
Primer jednog konfiguracionog fajla je dat u okviru slike \ref{code:primer-konfiguracije-toka-podataka}.
\begin{listing}[H]
\begin{minted}[frame=single,
               framesep=3mm,
               linenos=true,
               xleftmargin=21pt,
               tabsize=4]{js}
input {
    http {
        port => 3333
        tags => gateway
    }
}
filter {
    . . . 
}
output {
    . . .
}
\end{minted}
\caption{\textit{Primer konfiguracije toka podataka}}
\label{code:primer-konfiguracije-toka-podataka}
\end{listing}

\par
Dodaci mogu zahtevati vrednosti za određena podešavanja. Tipovi vrednosti koje postoje u okviru konfiguracije su:
\begin{itemize}
    \item \textbf{Lista} – definiše se uglastim zagradama, dok su elementi unutar zagrada odvojeni zarezima,
    \item \textbf{Logička vrednost} – definiše se vrednostima \mintinline{js}{true} i \mintinline{js}{false},
    \item \textbf{Bajt} – definiše se kao string koji sadrži broj bajtova sa mernom jedinicom, bilo da je u pitanju SI (osnova 1000), ili binarna (osnova 1024) merna jedinica. Polje nije osetljivo na velika slova \textit{(eng. case-insensitive)} i prepoznaje razmak između vrednosti i jedinice. Ukoliko se ne napiše jedinica, već samo celobrojna vrednost, podrazumeva se da vrednost predstavlja egzaktan broj bajtova,
    \item \textbf{Kodek} – definiše ime jednog od \textit{\textbf{Logstash}} kodeka, koji se može naznačiti i u ulaznim i u izlaznim etapama. Kada se nađe u ulaznoj etapi, predstavlja način dekodiranja podataka koji ulaze u tok, dok na izlazu predstavljaju način kodiranja u određeni format,
    \item \textbf{Heš mapa} – definiše kolekciju ključ-vrednost elemenata u formatu \textit{"polje" => "vrednost"}, gde se elementi razdvajaju blanko znacima, ne zapetama kao kod listi i nizova,
    \item \textbf{Broj} – obuhvata realne brojeve, odnosno \textit{integer} i \textit{float} vrednosti,
    \item \textbf{Lozinka} – string vrednost koja ne biva evidentirana,
    \item \textbf{URI} – string vrednost koja predstavlja identifikator resursa, može biti potpun ili parcijalni \textit{URL}. Ukoliko sadrži korisničko ime i lozinku, takođe se neće evidentirati,
    \item \textbf{Putanja} – definiše validnu putanju do resursa na sistemu,
    \item \textbf{String} – sekvenca karaktera, ukoliko sadrži blanko znake, neophodno je ograditi ga jednostrukim ili dvostrukim znakom navoda,
    \item \textbf{Referenca na polje} – predstavlja putanju do polja u događaju, ukoliko se radi o polju koje je u korenu događaja, onda se može koristiti notacija \mintinline{js}{[polje]} ili se izostaviti uglaste zagrade, polje. Ukoliko se referencira ugnježdeno polje, onda je neophodno koristiti notaciju sa uglastim zagradama i to po ugledu na oblik \mintinline{js}{[polje prvog nivoa][polje drugog nivoa]...[polje n-tog nivoa]}.
    \item \textbf{Komentari} se označavaju kao u programskim jezicima \textit{Perl}, \textit{Ruby} i \textit{Python}, odnosno počinju znakom taraba (\#).
\end{itemize}

\par
\textbf{Uslovne strukture} \textit{(eng. conditionals)} su podržane, imaju poznatu strukturu kao i u ostalim programskim jezicima \mintinline{python}{if, else if, else}, a koriste se za dodatno kontrolisanje filter i izlazne etape toka podataka. Primer strukture je dat u nastavku \ref{code:uslovne-strukture-u-konfiguracionom-fajlu-toka-podataka}.
\begin{listing}[H]
\begin{minted}[frame=single,
               framesep=3mm,
               linenos=true,
               xleftmargin=21pt,
               tabsize=4]{python}
if EXPRESSION {
    ...
} else if EXPRESSION {
    ...
} else {
    ...
}
\end{minted}
\caption{\textit{Uslovne strukture u konfiguracionom fajlu toka podataka}}
\label{code:uslovne-strukture-u-konfiguracionom-fajlu-toka-podataka}
\end{listing}

\par
\textbf{Izraz} \textit{(eng. expression)} predstavlja logički izraz koji se svodi na jednu od logičkih vrednosti \mintinline{js}{true}, ili \mintinline{js}{false}. Ukoliko izraz sadrži referencu na polje, vrednost izraza će biti \mintinline{js}{false} ukoliko: polje ne postoji, polje ima nedefinisanu vrednost \mintinline{js}{null} ili postoji i ima vrednost \mintinline{js}{false}. Podržani operatori u okviru jezika podrazumevaju:
\begin{itemize}
\item Operatore poređenja:
\begin{itemize}
\item[o] Jednakost: \textbf{==}, \textbf{!=}, \textbf{<}, \textbf{>}, \textbf{<=}, \textbf{>=}
\item[o] Regularni izraz: \textbf{=~}, \textbf{!~}
\item[o] Provera sadržaja: \textbf{in}, \textbf{not in}
\end{itemize}
\item Logičke operatore: \textbf{and}, \textbf{or}, \textbf{nand}, \textbf{xor}
\item Unarni operator: \textbf{!}
\end{itemize}

\par
Konkretan primer uslovne strukture sa izrazima je dat na slici \ref{code:primer-izraza-u-okviru-uslovnih-struktura}.
\begin{listing}[H]
\begin{minted}[frame=single,
               framesep=3mm,
               linenos=true,
               xleftmargin=21pt,
               tabsize=4,
               fontsize={\fontsize{10}{10}\selectfont}]{python}
filter {
    if [fooVal] in ["test", "debug"] and ([logLevel] != "debug") {
        mutate { add_tag => "testing" }
    }
    if [foo] =~ "[0-9]+" {
        mutate { add_tag => "contains numbers" }
    }
    if "%{+HH}" < "16" {
      mutate { add_tag => "Before 16h" }
    }    
}
output {
    if "testing" not in [tags] {
        elasticsearch{
            ...
        }
    }
}
\end{minted}
\caption{\textit{Primer izraza u okviru uslovnih struktura}}
\label{code:primer-izraza-u-okviru-uslovnih-struktura}
\end{listing}

\par
Svakom događaju koji se generiše u ulaznoj etapi obrade se pridružuju polja koja bliže opisuju događaj. Ova polja su rezervisana, njihov tip zavisi od implementacije samog ulaznog dodatka koji ih generiše, a namenjeni su boljoj kontroli toka obrade:
\begin{itemize}
    \item \textbf{@metadata} – heš mapa namenjena za skladištenje pomoćnih podataka u toku obrade. Ovo polje je dostupno u filter i izlaznoj etapi, međutim, podrazumevano se ne šalje na izlazne dodatke.
    \item \textbf{@timestamp} – predstavlja vremenski trenutak u kom je generisan događaj. Iz ovog polja se čitaju podaci koje koristi \textit{sprintf} format, koji omogućava referenciranje vrednosti iz ostalih polja događaja.
    \item \textbf{@version} – predstavlja verziju dodatka koji je generisao događaj.
    \item \textbf{tags} – niz vrednosti koji se koriste radi bližeg opisivanja događaja.
\end{itemize}

\subsection{Dodaci toka obrade}
\textbf{Dodaci toka obrade} \textit{(eng. pipeline plugins)} su programi napisani u programskom jeziku \textit{Java} ili \textit{Ruby}, koji predstavljaju alat za olakšavanje kontrole samog toka obrade, od ulaza, preko obrade, do izlaza samih podataka. Ove programe može napisati bilo ko u programskom jeziku \textit{Java} ili \textit{Ruby} i publikovati ih na javni \textit{Github} repozitorijum sa kog ostali korisnici mogu da ih povlače i koriste u svojim sistemima. Samim tim, \textit{\textbf{Logstash}} ima jako veliki skup dodataka koji mogu biti korišćeni, dok će ovaj rad nadalje obraditi samo one koji su neophodni za razumevanje sistema koji se implementira.

\par
\textbf{TCP ulazni dodatak} \cite{tcp-plugin} vrši čitanje poruka poslatih preko mreže, tako što osluškuje poruke TCP protokola, generišući odgovarajuće događaje za tok podataka u kom je konfigurisan. 

\par
\textbf{Grok filter dodatak} \cite{grok-plugin} vrši parsiranje proizvoljnog teksta i njegovo prevođenje u strukturni podatak, nad kojim se mogu pisati upiti. Funkcioniše tako što koristi regularne izraze za pronalaženje šablona u okviru tekstualnih polja. Sintaksa koju prati ovaj dodatak je \textbf{\%{SINTAKSA:SEMANTIKA}}. Sintaksa predstavlja ime unapred definisanog šablona regularnog izraza koji se koristi za identifikovanje dela teksta, dok semantika predstavlja identifikator pronađenog teksta koji se dalje može koristiti u toku podataka. Pored unapred definisanih šablona, mogu se direktno pisati regularni izrazi za pronalaženje šablona. Sintaksa regularnih izraza koja se koristi je \textit{Onigurama} \cite{onigurama}, koja ima izgled \textit{(?<identifikator polja>šablon)}. Još jedan način jeste kreiranje sopstvenih šablona u odvojenim fajlovima (putanje do fajlova se definišu u konfiguraciji dodatka), dok novo-definisani šablon ima izgled \textit{IME\_ŠABLONA ŠABLON}.

\par
\textbf{Mutate filter dodatak} \cite{mutate-plugin} omogućava generalizaciju mutacija nad poljima događaja. Moguće su operacije: preimenovanja, zamene, modifikacije polja u okviru događaja i slično. Redosled operacija u okviru dodatka je unapred ustanovljen, odnosno ne prati redosled navođenja operacija u okviru konfiguracije. Ukoliko je neophodno izvršiti više operacija koje ne prate ovakav redosled, neophodno je definisati više \textit{mutate} filter dodataka u odgovarajućem redosledu. 

\par
\textbf{Geoip filter dodatak} \cite{geoip-plugin} nadograđuje događaje informacijama o geo-prostornoj lokaciji prosleđene \textit{IP} adrese. Ovaj dodatak je zasnovan na podacima iz \textit{MaxMind GeoLite2} baze podataka \cite{maxmind-geolite}, koja dolazi kao podrazumevana baza dodatka. Moguće je kroz opcije konfigurisati bazu podataka, odnosno promeniti podrazumevanu. Primer odgovora koji se dobija za prosleđenu \textit{IP} adresu dat je u okviru slike \ref{code:primer-podataka-koje-pruza-geoip-filter-dodatak-za-prosledjenu-ip-adresu}.

\begin{listing}[H]
\begin{minted}[frame=single,
               framesep=3mm,
               linenos=true,
               xleftmargin=21pt,
               tabsize=4]{js}
{
    "ip": "12.34.56.78",
    "geo": {
        "city_name": "Seattle",
        "country_name": "United States",
        "cotinent_code": "NA",
        "continent_name": "North America",
        "country_iso_code": "US",
        "postal_code": 98106,
        "region_name": "Washington",
        "region_code": "WA",
        "region_iso_code": "US-WA",
        "timezone": "America/Los_Angeles",
        "location": {
            "lat": 47.6062,
            "lon": -122.3321,
        }
    },
    "domain": "example.com",
    "asn": {
        "number": 98765,
        "organization": {
            "name": "Elastic, NV"
        }
    },
    "mmdb": {
        "isp": "InterLi Supra LLC",
        "dma_code": 819,
        "organization": "Elastic, NV"
    }
}
\end{minted}
\caption{\textit{Primer podataka koje pruža geoip filter dodatak za prosleđenu IP adresu}}
\label{code:primer-podataka-koje-pruza-geoip-filter-dodatak-za-prosledjenu-ip-adresu}
\end{listing}

\par
\textbf{Http filter dodatak} \cite{http-plugin} omogućava integraciju sa eksternim \textit{Web} servisima ili \textit{REST API}-jima, tako što osposobljava \textit{\textbf{Logstash}} da šalje zahteve podesive strukture na odgovarajuće krajnje tačke \textit{(eng. endpoint)}, kako bi obogatio događaj koji se obrađuje dodatnim informacijama. Parametar koji definiše zaglavlja zahteva \textit{(eng. headers)} koji se šalju bivaju definisana u okviru \textbf{@metadata} polja, kako ne bi okupirali destinacije na izlazu. Ovo je vrlo podesiv dodatak i ima veliki broj opcija za konfiguraciju.

\par
\textbf{Dns filter dodatak} \cite{dns-plugin} pruža pretragu stavki, kanoničkih imena \textit{(eng. CNAME)}, ili stavki adresa \textit{(eng. A record)}, odnosno ukoliko je u pitanju inverzni \textit{DNS}, pretražuju se pointer stavke \textit{(eng. pointer record, PTR)}. Bitno je napomenuti da ovaj dodatak može da obradi samo jednu stavku, dakle za veći broj stavki je potrebno više puta iskoristiti ovaj dodatak, s tim što ovakvo podešavanje može drastično da uspori sistem. 

\par
\textbf{Elasticseach izlazni dodatak} \cite{elasticsearch-plugin} omogućava skladištenje vremenskih serija podataka, poput podataka evidencije, događaja i metrike, kao i podataka čije skladištenje nije bazirano na vremenskim intervalima, direktno u \textit{Elasticsearch}. S obzirom da se radi o dodatku koji radi sa osnovnim proizvodom kompanije, ovaj dodatak je jako fleksibilan i može kontrolisati razne segmente vezane za operacije u \textit{Elasticsearch}-u. Najbitnije opcije u okviru dodatka, koje se vezuju za temu ovog rada podrazumevaju:
\begin{itemize}
    \item \textbf{action} – enumeracija(\textit{index, delete, create, update}) koja definiše koja vrsta operacije biva izvršavana. Podrazumevana vrednost za vremenske serije podataka, odnosno tokove podataka je \textit{create}, dok je \textit{index} podrazumevana vrednost za podatke koji se ne vezuju za vremenske intervale.
    \item \textbf{index} – string koji predstavlja indeks u koji se upisuju podaci obrađeni u okviru toka podataka
    \item \textbf{hosts} – uri tip vrednosti, koji predstavlja jednu ili više adresa čvorova na kojima je pokrenuta instanca \textit{Elasticsearch}-a. Podrazumevana vrednost je \mintinline{python}{[//127.0.0.1]}.
\end{itemize}