\section{\textit{\textbf{Kibana}}}
\textit{\textbf{Kibana}} je aplikacija otvorenog koda koja pruža pregledan korisnički interfejs ka \textit{Elastic stack}-u, uz mogućnosti pretraživanja, vizuelizacije i analize podataka indeksiranih \textit{Elasticsearch}-om. Takođe poznata i kao alat za kreiranje raznovrsnih grafikona \textit{(eng. charts)}, \textit{\textbf{Kibana}} omogućava nadgledanje, upravljanje i održavanje bezbednosti klastera u \textit{Elastic stack}-u. Prvi put je postala dostupna javnosti 2013. godine, dok je trenutna aktuelna verzija 8.6, na osnovu koje će biti opisani pojedini detalji same aplikacije \cite{elastic-kibana}.

\par
Glavna primena \textit{\textbf{Kibana}}-e podrazumeva: pretraživanje, vizualizaciju indeksiranih podataka u okviru \textit{Elasticsearch}-a, kao i analizu podataka kroz kreiranje grafikona, poput: poluga, pita, tabela, histograma i mapa. Komandna tabla \textit{(eng. dashboard)} kombinuje ove elemente na jedan pano i time omogućava analitički pregled podataka u realnom vremenu za razne slučajeve korišćenja, poput:
\begin{enumerate}
    \item analize podataka evidencije, 
    \item nadgledanja metričkih podataka infrastrukture i kontejnera, 
    \item vizuelizaciju geo-prostornih podataka, 
    \item analizu sigurnosnih podataka, 
    \item analizu podataka biznis logike.
\end{enumerate}

\subsection{Upitni jezik \textit{\textbf{Kibana}}-e}
\textit{\textbf{Kibana}} koristi specifični upitni jezik nazvan \textit{\textbf{Kibana Query Language}} \cite{kql}, ili skraćeno \textit{\textbf{KQL}}, kojim je omogućeno jednostavno filtriranje podataka u svrhu vizualizacije. Ovaj jezik se razlikuje od standardnog \textit{Lucene }upitnog jezika, jer ne omogućava pretragu uz pomoć regularnih izraza ili takozvanu pomućenu \textit{(eng. fuzzy)} pretragu podataka, međutim omogućena je pretraga ugnježdenih polja kao i takozvanih skriptovanih polja \textit{(eng. scripted fields)}.

\par
U okviru jezika se mogu identifikovati podvrste upitnog jezika: \textbf{upiti termina}, \textbf{upiti Bulove algebre}, \textbf{opsežni upiti}, \textbf{upiti koji koriste džokere} i \textbf{upiti nad ugnježdenim poljima}.

\par
\textbf{Upiti termina} \textit{(eng. terms query)}, koji koriste egzaktan stil pretrage. Sintaksa ovog upita ima oblik \textit{<putanja do polja> : <list termina>}, gde putanja do polja predstavlja ugnježdena polja počevši od korena dokumenta, sve do polja koje se pretražuje, nivoi su razdvojeni tačkom. Lista termina predstavlja prihvatljive vrednosti u okviru navedenog polja, dok se različiti termini odvajaju razmakom, a ukoliko se pretražuju egzaktne fraze, koriste se znaci navođenja. Putanja do polja i lista termina se odvajaju specijalnim karakterom dvotačka, na koji može da se gleda kao na operator \textit{in}.

\par
\textbf{Upiti Bulove algebre} \textit{(eng. boolean queries)} predstavljaju kombinovanje prethodno navedene podvrste upita logičkim operatorima: \textbf{or}, \textbf{and} i \textbf{not}. Po pravilu, \textbf{and} ima viši prioritet od operatora \textbf{or}, ukoliko je neophodna drugačija logika, koriste se zagrade.

\par
\textbf{Opsežni upiti} \textit{(eng. range queries)} predstavljaju upite koji numeričke i vrednosti datuma navedenih polja porede operatorima jednakosti: \textbf{>}, \textbf{>=}, \textbf{<} i \textbf{<=}. Moguće je koristiti i matematičke izraze pri poređenju vrednosti, što je jako pogodno za datume.

\par
\textbf{Upiti koji koriste džokere} \textit{(eng. wildcard)}, koji je označen znakom \textbf{*} i može biti ili u delu putanje polja, čime se pokriva veći broj polja, ili u okviru vrednosti koje se traže u poljima i time pokriva veći spektar vrednosti, jer menja odgovarajući deo bilo kojom sekvencom bilo koje dužine.

\par
\textbf{Upiti nad ugnježdenim poljima} \textit{(eng. nested field queries)} omogućavaju dva pristupa u filtriranju ugnježdenih dokumenata:
\begin{itemize}
    \item Filtriranje jednog dokumenta na osnovu određenih delova upita\\
\textit{<polje>:\{ <ugnježdeno polje 1>:<izraz 1> and <ugnježdeno polje 2>:<izraz 2> \}}
    \item Filtriranje više dokumenata na osnovu određenih delova upita \\
\textit{<polje>:\{ <ugnježdeno polje 1>:<izraz 1> \} and <polje>:\{ <ugnježdeno polje 2>:<izraz 2> }\}
\end{itemize}