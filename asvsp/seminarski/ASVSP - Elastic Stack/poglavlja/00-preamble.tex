\usepackage[table]{xcolor}
\usepackage{svg}
% фонтови и језик
% fontspec docs: shorturl.at/ouI26
\usepackage{fontspec}

% polyglossia docs: shorturl.at/gELX9
\usepackage{polyglossia}
\newfontfamily\serbainlatinfont{Times New Roman}
\setmainlanguage[Script=latin]{serbian}
\setotherlanguage{english}
\setlength{\parskip}{\baselineskip}%

% вербатим линкови (веб адресе, имејлови, релативне адресе, итд.)
% url docs: shorturl.at/iktM4
\usepackage{url}

% хиперлинкови
% hyperref docs: shorturl.at/fmHPW
\usepackage{hyperref}

% приказивање математичких израза
% amsmath docs: shorturl.at/avJU3
\usepackage{amsmath}
\usepackage{mathtools}
\DeclarePairedDelimiter{\ceil}{\lceil}{\rceil}

% напредни пакет за исцрватање графика коришћењем команде \includegraphics
% graphicx docs: shorturl.at/diHUW
\usepackage{graphicx}
\graphicspath{{images/}}    % коренски директоријум слика, сваки пут када се користи includegraphics команда путања која се задаје треба да буде релативна у односу на овај директоријум

% подешавања маргина
\usepackage{vmargin}
\setmarginsrb{3 cm}{2.5 cm}{3 cm}{2.5 cm}{1 cm}{1.5 cm}{1 cm}{1.5 cm}

% исцртавање програмског кода
\usepackage{minted}

% подешавање размака линија текста
\usepackage{setspace}

% цртање табела
\usepackage{booktabs}

% додатни макрои за израду рада
\usepackage{rsvp}

%\usepackage[Glenn]{fncychap}
\usepackage{tikzit}

\usepackage{algorithm}
\usepackage[noend]{algpseudocode}

\makeatletter
\def\BState{\State\hskip-\ALG@thistlm}
\makeatother
\floatname{algorithm}{Algoritam}

\usepackage{nicematrix}

\usepackage{pgfplots}
\pgfplotsset{compat=1.18}

%\renewcommand{\caption}[1]{\caption{\textit{#1}}}

%%%%%%%%%%%%%%%%%%%%%%%%%%%%%%%%%%%%%%%%%%%%%%%%%%%%%%%%%

\usepackage{fancyhdr}

\makeatletter
\let\thetitle\@title
\let\theauthor\@author
\let\thedate\@date
\let\theindex\studentindex
\makeatother

\pagestyle{fancy}
\fancyhf{}
\rhead{\theauthor}
\lhead{\thetitle}
\cfoot{\thepage}


\hypersetup{
    colorlinks=true,
    allcolors={blue!60!black}
}
\newcolumntype{B}{>{\columncolor{blue!10}} c}
\definecolor{light blue}{HTML}{D5EBFF}
\definecolor{dark blue}{HTML}{00008B}
\newcommand{\textbw}[1]{\textbf{\textcolor{white}{#1}}}